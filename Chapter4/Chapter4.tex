\section{Economic Analysis}

\subsection{Project description}
inClass is a project developed for a specific target group, which are people involved in academic structures, like universities. The main task of the application is to ease the access to the current studying timetable by generating the individual version for each user.  There are three types of users involved: administrators (they are in charge of introducing data into the system), students and teachers (they are able to access the data from the system through account registration). Given that nowadays the technological progress is exponentially growing, the fact that \text{inClass} is a web application represents a major advantage for its users. As a matter of principle, universities already guarantee internet access on their territories, meaning that now it is possible to access inClass from a smartphone at any moment of time. Having it available online gives flexibility to teachers and students during classes and breaks. Instead of visiting the general timetable from the hole of the university, it is enough  to navigate on the internet and be aware of where the next course should take place.

It is worth mentioning that there is no alternative application in our country that would offer such kind of service. Similar projects are already implemented in countries abroad and represent a big step for transferring the manual administration system to the computerized one. Having such an application integrated in the university's structure would bring benefits to all implied parts as it would be out of charge for users and easy to maintain for administrators. The only extra job that would allow avoiding the old habits is maintenance of the data by updating it once in a semester, when the timetable changes.
 
In order to proceed to  the implementation of the system, it is necessary to think about a project budget, which offers a scientific and justifiable economical point of view over the system itself. In order to decide if the product is viable on the national economy, it is expected to be elaborated an initial overview of the costs and incomes after launching on the market. That's why, economic analysis is one of the first steps, meant to motivate, create and implement the given informational system, whose final result would be a general overview on the elaboration of the system. In this chapter, there are going to be analyzed all the expenses necessary to elaborate the project, starting from the materials/non-materials used for long term, time schedule establishment and indirect expenses. Then there will be computed the days necessary to elaborate the project, the team involved and the salary for each person. After computing these indicators and having a synopsis on the subject in question, will be much easier to predict the next steps to be done in order to have a profit.

\newpage
\subsection{Project time schedule}
Given the fact that the system in discussion can be considered a complex one, it is expected to plan a time schedule first. For better results and efficient outcome, Agile software development will be applied. It promotes adaptive planning, evolutionary development, early delivery and continuous improvement. It also encourages rapid and flexible response to change. It is an iterative and incremental process that goes through 5 stages: planning, research, development, testing, and deployment. The benefit of using Agile is that a prototype is created in the early stage of development and all the future modifications can be treated as improvements for the base application.

\addtocontents{toc}{\protect\newpage}
\subsubsection{SWOT Analysis}
Performing the SWOT analysis over a system gives a brief overview about expectations or possible problems that can appear during the lifetime of the system.It is mandatory to identify the SWOTs because they can inform later steps in planning to achieve the expected objectives. In Table \ref{table:swot}  it is represented the strategic planning method, used to evaluate Strength, Weaknesses, Opportunities and Threads that can involve the given system.


\begin{table}[H]
\centering
\caption{SWOT analysis}
\begin{tabular}{|l|l|}
\hline
Strengths & Weaknesses
\\ \hline
\begin{minipage}{3.1in}
	\vspace{.5cm}
	\begin{itemize}[leftmargin=*]
	\item No competitors on local market;
	\item Accessible and portable; 
	\item It is expected to be highly profitable;
	\item Easy maintainable;
	\end{itemize}
	\vspace{.5cm}
\end{minipage}
&
\begin{minipage}{3.1in}
	\begin{itemize}[leftmargin=*]
	\item Systematic database updates needed;
	\item Require internet connection in order to access;
	\item Not fully responsive on smartphones;
	\end{itemize}
\end{minipage}

\\ \hline
Opportunities & Threats 
\\ \hline
\begin{minipage}{3.1in}
	\vspace{.5cm}
	\begin{itemize}[leftmargin=*]
	\item[--] Ensures with access 2 types of users (teachers, students);
	\item[--] Nice solution for personal time management;
	\end{itemize}
	\vspace{.5cm}
\end{minipage}
&
\begin{minipage}{3.1in}
	\begin{itemize}[leftmargin=*]
	\item Heavy loads at each database update;
	\item Account registration for teachers not fully secured;
\end{itemize}
\end{minipage}
\\ \hline
\end{tabular}
\label{table:swot}
\end{table}

Referring to the SWOT analysis as the basis of the promotion/development of any product, each company has the opportunity to foresee the possible profit as well as possible wastage. One of the most dangerous factors that can be prevented with SWOT is the risk of concurrency, which also has an important role in market development and the increase of product's quality. To predict other negative effects, any company must have a well-designed plan, like having monthly statistics, which would contribute at diminishing economical threats.

\subsubsection{Defining objectives}
The main objective of the project is to improve the individual time management both for teachers and students. Also, as a new product on the national market, its purpose is to spread among universities as an administration tool for semester timetables. The application has several development targets, which would enrich the application with even more functionality for personal usage and monitoring of current study situation. As the application is provided as a free product, it is expected to gain users among all academical infrastructures.

\subsubsection{Time schedule establishment}
There are five main steps to be accomplished in order to reach a final result for the project in discussion:
\begin{itemize}
  \item Planning (form some ideas about the design of the project, build some sketches and diagrams, determine amount of needed resources)
  \item Researching (analyze the market, the target group for which the project is intended, learn technologies for further use)
  \item Developing (create the actual system, adjust requirements to the used technologies and frameworks)
  \item Testing (perform debugging in order to ensure the well behavior of the system and integrity of functionality )
  \item Deployment (make the application available for users)
\end{itemize}

The key to a well-scheduled plan, that would lead to the in-time accomplishment of the project, is the proper subdivision of the workload, according to the given resources. Meaning that, every step enumerated above, should be scheduled for a finite period of time, judging by the level of difficulty implied. The planning period should be considered a flexible one, since at this moment some general ideas on the project have to be discussed. Changes are allowed at this step, since requirements are still established. For the researching period, it is important to be open to new ideas for consideration, in order to have a better outcome. The development period should be the most accurate one and should be divided in sprints, for better monitoring of the workflow. In the table bellow are represented all the general steps involving the development of the project. The following adnotations were used: PM –
project manager, SA – system architect, SM – sales manager, D – developer.


\begin{table}[H]
\centering
\caption{Time schedule}
\resizebox{\textwidth}{!}{\begin{tabular}{| c | c | c | c |}
\hline
\textbf{Nr} & \textbf{Activity Name} & \textbf{Duration (days)} & \textbf{People involved} \\
\hline
1 & Analyze requirements and activities schedule & 7 & PM, SA, SM, D\\
\hline
2 &  Perform market analysis & 5 & PM, SM\\
\hline
3 & Establish functional features & 14 &  PM, SA\\
\hline
4 & Elaborate Use Case Diagrams & 5 & PM, SA, D\\
\hline
5 & Implement database design & 7 &  PM, SA, D\\
\hline
6 & Implement website design & 7 & PM, SA, D\\
\hline
7 & Create back-end functionality & 28 & D\\
\hline
8 & Test back-end functionality & 4 & D\\
\hline
9 & Create front-end functionality & 21 & D\\
\hline
10 & Test entire project& 7 & PM, SA, D, SM\\
\hline
11 & Validate results & 3 & PM, SA, D, SM\\
\hline
12 & Write documentation & 7 & D\\
\hline
13 & Deploy the system & 2 & PM, SA, D\\
\hline
14 & Commercialize the product & 3 & SM\\
\hline
15 & Total time to finish the system & 120 &\\
\hline
\end{tabular}}
\label{table:schedule}
\end{table}
 
 As a generalization, Table \ref{table:schedule} represents a sketch for the first iteration of the project. Each activity was evaluated with a time stamp and responsible employees were assigned. For the completion of the purposed system including 14 activities was estimated a total amount of time including 120 working days. Bellow are indicated the amount of spent time for each individual:
 \begin{itemize}
\item PM: 50 days; 
\item SA:  45 days; 
\item SM: 13 days;
\item D: 101 days.
 \end{itemize}

\subsection{Economic motivation}
It is actual in economy to bring economical proofs for the IT projects, basing on the specific of the concurrences in economic relationships, which suppose a wide research space. In the conditions of a low degree of determination of the marketing environment, of high prices' volatility, decreased degree of prognoses depth, a common business-plan doesn't allow the exact foreseeing of the final results of the business. In this context, one of the basic instruments is choosing the methods, the right positions and index for the economical proofs.
Realization of this goal conditions a large number of scientifically research, subordinated to the primary goal and formulated by means of the following objectives:
\begin{itemize}
\item Studying the theoretical and methodical aspects of the business-planning in the conditions of the concurrency on the market;
\item  Systematization, determining the methodology and specifying the index for the economical proof of the business-plans in IT;
\item  Study and analysis of the actual practice of economical proofs of the business-plans for IT in Republic of Moldova;
\item  Developing methodological concepts of the proofs of the decision of investment in the conditions of risk and incertitude;
\item  Studying the evaluation criteria of the business-projects' efficiency and elaboration of a mechanism of complex evaluation of these.
\end{itemize}

\subsubsection{Tangible and intangible asset expenses}
Expenses and initial budget are the ones that sharpen from the beginning the project itself, by imposing some limitations on the complexity of the system and defining its boundaries. In this section, an evaluation of the necessary amount of money will be computed, in order to ensure the correct administration of financial resources. At the beginning, in Table \ref{table:tangible_assets}, will be listed all the tangible assets used for the project. Tangible assets are defined as any assets that have a physical form. 

\begin{table}[H]
\centering
\caption{Tangible assets expenses}
\resizebox{\textwidth}{!}{\begin{tabular}{| c | c | c | c | c | c |}
\hline
\textbf{Material} &\textbf{Specification} & \textbf{Measurement unit} & \textbf{Price per unit (MDL)} & \textbf{Quantity} & \textbf{Sum (MDL)}\\
\hline
MacBook Pro & i5 processor & Unit & 24000 & 1 &  \multicolumn{1}{r|}{24000}\\
\hline
\multicolumn{5}{|r|}{Total} & \multicolumn{1}{r|}{24000}\\
\hline
\end{tabular}}
\label{table:tangible_assets}
\end{table}

The Table \ref{table:intangible_assets} presents the intangible assets, used for the project. Intangible assets are all those that do not posses a physical form. In this case, the discussion is about the \textbf{software} needed to build the application. For the design of UML diagrams, \emph{Enterprise} Architect was used.

\begin{table}[H]
\centering
\caption{Intangible asset expenses}
\resizebox{\textwidth}{!}{\begin{tabular}{| c | c | c | c | c | c |}
\hline
\textbf{Material} & \textbf{Specification} & \textbf{Measurement unit} & \textbf{Price per unit (MDL)} & \textbf{Quantity} & \textbf{Sum (MDL)} \\
\hline
License & Enterprise Architect Desktop Edition License & Unit & 1800 & 3 & \multicolumn{1}{r|}{5400} \\
\hline
License & Sublime Text 3 & Unit & 1200 & 3 & \multicolumn{1}{r|}{3600} \\
\hline
\multicolumn{5}{|r|}{Total} & \multicolumn{1}{r|}{9000}\\
\hline
\end{tabular}}
\label{table:intangible_assets}
\end{table}

Also, additional expenses represent the direct expenses, used for work facilitation during project development. These costs cannot be included in any of the previous tables, because their values aren't included directly into the budget of the project and they have to be mentioned out of the topic. The Table \ref{table:direct_expenses} emphases the direct expenses:

\begin{table}[H]
\centering
\caption{Direct expenses}
\resizebox{\textwidth}{!}{\begin{tabular}{| c | c | c | c | c | r |}
\hline
\textbf{Material} & \textbf{Specification} & \textbf{Measurement unit} & \textbf{Price per unit (MDL)} & \textbf{Quantity} & \multicolumn{1}{c |}{\textbf{Sum (MDL)}}\\
\hline
Whiteboard & Universal Dry Erase Board & Unit & 700 & 1 & 700 \\
\hline
Paper & A4 & 100 sheets & 60 & 1 & 60 \\
\hline
Pen & Blue pen & Unit & 5 & 10 & 50 \\
\hline
\multicolumn{5}{|r|}{Total} & 810 \\
\hline
\end{tabular}}
\label{table:direct_expenses}
\end{table}

Given all the raw data, a total amount of direct expenses can be calculated: 
\begin{equation}
 T_{e} = 24000 + 9000 + 810 = 33810
\end{equation}

%\newpage
\subsubsection{Salary expenses}
In this compartment will be discussed the remuneration of each employee during the development of the project. It is assumed that each member of the team has the salary listed bellow:

\begin{itemize}
\item Project Manager – 500 MDL
\item System Architect – 580 MDL
\item Sales Manager – 350 MDL
\item Developer – 400 MDL
\end{itemize}

After the closing of the project, some statistics regarding financial costs for employee remuneration can be presented in Table \ref{table:salaries}:

\begin{table}[H]
\centering
\caption{Salary expenses}
\resizebox{\textwidth}{!}{\begin{tabular}{| c | c | c | r |}

\hline
\textbf{Employee} & \textbf{Working days} & \textbf{Salary per day (MDL)} & \multicolumn{1}{c|}{\textbf{Salary fund (MDL)}}\\
\hline
Project Manager & 50 & 500 & 25000 \\
\hline 
System Architect & 45 & 580 & 26100\\
\hline
Sales Manager & 13 & 350 & 4550\\
\hline
Developer & 101 & 400 & 40400\\
\hline
\multicolumn{3}{|r|}{Total} & 96050\\
\hline
\end{tabular}}
\label{table:salaries}
\end{table}

Besides salaries, the social service fund also retrieves a part of the money, constituting $23\%$ of total  salary. Also, there is the medical insurance fund, constituting $4,5\%$ of the same sum. The next step is to compute the social service fund, according to the relation \eqref{eq:sf} :

\begin{equation}
\label{eq:sf}
\begin{split}
 FS &= F_{re} \cdot T_{fs} \\
    &= 96050 \cdot 0.23  \\
    &= 22092,
\end{split}
\end{equation}
\noindent
where $FS$ is the salary expense, $F_{re}$ is the salary expense fund and $T_{fs}$ is the social service tax approved each year. The medical insurance fund is computed as:

\begin{equation}
\begin{split}
 MI &= F_{re} \cdot T_{mi}\\ 
    &= 96050 \cdot 0.045\\ 
    &= 4322,
 \end{split}
\end{equation}

\noindent
where $T_{mi}$ is the mandatory medical insurance tax approved each year by law of medical insurance.

Now, the total work expense fund is calculated as sum of the previous computed indicators:

\begin{equation}
\begin{split}
 WEF &= F_{re} + FS + MI\\
     &= 96050 + 22092 + 4322\\
     &= 122464,
\end{split}
\end{equation}

\noindent
where $WEF$ is the work expense fund, FS is the social fund and MI is the medical insurance fund. The final indicator shows the total work expense fund.

\subsection{Individual person salary}
Having the total work expense fund computed, it is necessary to determine the net salary for the developer. Considering the developer's salary of 400 MDL per day and there is a totally 120 working days for accomplishing the project, so the gross salary that the developer gets is:

\begin{equation}
 GS = 400 \cdot 120 = 48000,
\end{equation}

\noindent where $GS$ is the gross salary computed in MDL.

Social fund tax this year represents $6\%$, so the amount that should be tax paid in MDL represents

\begin{equation}
 SF = 48000 \cdot 0.06 = 2880.
\end{equation}

Medical insurance tax represents $3.5\%$ and gives the following result

\begin{equation}
 MIF = 48000 \cdot 0.045 = 2160.
\end{equation}

In order to proceed with income tax computations, it is necessary to calculate the amount of taxed salary.

\begin{equation}
\begin{split}
 TS &= GS - SF - MIF - PE \\
              &= 48000 -2880 - 2160 - 10128- \\ 
              &= 32832,
\end{split}
\end{equation}

\noindent
where $TS$ is the taxed salary, $GS$ -- gross salary, $SF$ -- social fund, $PE$ -- personal exemption, which this year is approved to be $10128$.

The last but not the least thing to be computed is the total income tax, which is $7\%$ for income under 29640 MDL and $18\%$ for income over 29640 MDL.

\begin{equation}
\begin{split}
 IT &= TS - ST \\
      &= 29640 \cdot 0.07 + (32832 - 29640) \cdot 0.18 \\
      & = 2074.8 + 574.6 = 2649.4,
 \end{split}
\end{equation}

\noindent
where $IT$ is the income tax, $TS$ -- the taxed salary and $ST$ -- the salary tax. 

With all this now it is possible to find out what's going to be the net income.

\begin{equation}
\begin{split}
 NS &= GS - IT - SF - MIF \\
            &= 48000 - 2649.4 - 2880 - 2160 \\
            &= 40310.6,
\end{split}
\end{equation}

\noindent
where $NS$ is the net salary, $GS$ -- gross salary, $IT$ -- income tax, $SF$ -- social fund, $MIF$ -- medical insurance fund.

\subsubsection{Indirect expenses}
Other expenses involved in the completion of the project involves production consumption. In the Table \ref{table:indirect_expenses} are mentioned expenses like public transport, electricity, internet access and office water. 

\begin{table}[H]
\centering
\caption{Indirect expenses}
\resizebox{\textwidth}{!}{\begin{tabular}{| c | c | c | c | c | r |}
\hline
\textbf{Material} & \textbf{Specification} & \textbf{Measurement unit} & \textbf{Price per unit (MDL)} & \textbf{Quantity} & \multicolumn{1}{c |}{\textbf{Sum (MDL)}}\\
\hline
Internet & Moldtelecom & Pack & 200.00 & 4 & 800 \\
\hline
Transport & Public bus & Trip & 2.00 & 240 & 480\\
\hline
Water & Apa-Canal Chisinau & m3 & 9.2 & 20 &184\\
\hline
Electricity & Union Fenosa & KWh & 2,16 & 500 & 1080\\
\hline
\multicolumn{5}{|r|}{Total} & 2544 \\
\hline
\end{tabular}}
\label{table:indirect_expenses}
\end{table}

\subsubsection{Wear and depreciation}
When speaking about economic analysis, it is vital to understand the influence of time over the product. Usually, it happens that a depreciation in value can appear, that's why it is a value that should be computed and take into account. As a tuple, with the depreciation, the wear will be calculated. Depression will be computed uniformly for the whole project duration, so that there are no accountancy issues. As a matter of fact, a business plan divided for 3 years should be compartmentalized into 3 uniform parts according to each year.
 

Normally wear is computed regarding to the type of asset. The computer mentioned above as tangible asset  can be used for a period of 3 years. Licenses will last for a single year. Straight line depreciation will be applied. First step is to sum up tangible and intangible assets, while the salvage costs of each of the items at the end of their period of use, has to be subtracted:

\begin{equation}
 \begin{split}
  TAV &= \sum_{} (AC - SV) \\
        &= (25200 - 5000) + (2880 - 1000) \\
        &= 22080,
 \end{split}
\end{equation}

\noindent
where $TAV$ is the total assets value, $AC$ -- assets cost, $SV$ -- salvage value.
%\newpage
 In order to get the yearly wear, divide total asset value by the period of use of assets, being 3 years.

\begin{equation} \label{eq:wear}
 \begin{split}
  W_y &= TAV / T_{use} \\
                &= 22080/3\\
                &= 7360,
 \end{split}
\end{equation}

\noindent
where $W_y$ is the wear per year, $TAV$ -- total assets value, $T_{use}$ -- period of use. Relation \eqref{eq:wear} included tangible assets which will last for 5 years and intangible assets which last only one year. The initial value of assets in MDL was

\begin{equation}
 \begin{split}
  W &= W_y / D_y \cdot T_p\\
                   &= 7360  / 365  \cdot 120 \\
                   &= 2420,
 \end{split}
\end{equation}

\subsubsection{Product cost}

At this point, it is time to compute the product cost which includes direct and indirect expenses, salary expenses and wear expenses as shown in Table \ref{table:product_cost}.

\begin{table}[H]
\centering
\caption{Total Product Cost}
\begin{tabular}{| c | r | r |}

\hline
\textbf{Expense type} & \multicolumn{1}{c |}{\textbf{Sum (MDL)}} & \multicolumn{1}{c |}{\textbf{Percentage (\%)}}\\
\hline
Direct expenses & 810 & 0.62 \\
\hline
Intangible expenses & 3600 & 2.78 \\
\hline
Salary expenses & 122464 & 94.7 \\
\hline
Asset wear expenses & 2420 & 1.87 \\
\hline
\textbf{Total product cost} & \textbf{129294} & \textbf{100}\\
\hline
\end{tabular}
\label{table:product_cost}
\end{table}

\newpage
\subsubsection{Economic indicators and results}
At the moment, all the expenses for the development of the project were computed. Now is time to consider how the application should be included on the market. As mentioned above, the target group which is meaningful for the given project represent the academic infrastructures, including teachers and students. So, as starting investors, should be considered universities' administrations. There is no optimal price established for the initial project, so a indicator of 25\% on top of the production cost will be used, in order to determine what the real cost the application should have.

\begin{equation}
 \begin{split}
  GP &= C_{total} / N_{cs} + P_{p}\\
              &= 129294/100 + 0.25 \cdot 1292.94\\
              &= 1616,
 \end{split}
\end{equation}

\noindent
where $GP$ is the gross price, $C_{total}$ -- total product cost, $N_{cs}$ -- number of copies sold, $P_{p}$ -- chosen profit percentage. This is not the price of the end product, since it is necessary to add sales tax (VAT), which represents $20\%$ and is added to the gross price. 

\begin{equation}
 \begin{split}
  P_{sale} &= GP + TX_{sales}\\
              &= 1616 + 0.2 \cdot 1616 \\
              &= 1932,
 \end{split}
\end{equation}

\noindent
where $P_{sale}$ is the sale prices including VAT, $GP$ -- gross price, $TX_{sales}$ -- sales tax. The net income is computed by multiplying gross price and the number of expected copies to be sold, which will be

\begin{equation}
 \begin{split}
  I_{net} &= GP \cdot N_{cs}\\
              &= 1616  \cdot 100 \\
              &= 161600,
 \end{split}
\end{equation}

\noindent
where $I_{net}$ is the net income, $GP$ -- gross price, $N_{cs}$ -- number of copies sold. Moreover it is necessary to compute the gross and net profit. The indicators are $GPr$ -- gross profit and $NPr$ -- net profit.

\begin{equation}
 \begin{split}
  GPr &= I_{net} - C_{production}\\
              &= 161600 - 129294\\
              &= 32306\\
  NPr &= GPr - 12\% \\
             &= 32306 - 12\% \\
             &= 28429,
 \end{split}
\end{equation}

\noindent
where $I_{net}$ is the net income, $C_{production}$ -- cost of production. The profitability indicators are $C_{profit}$ -- cost profitability, $S_{profit}$ -- sales profitability computed in MDL.

\begin{equation}
 \begin{split}
  C_{profit} &= GPr / C_{production} \cdot 100\%\\
              &= 32306 / 129294 \cdot 100\% \\
              &= 24.9 \%\\
  S_{profit} &= GPr / I_{net} \cdot 100\% \\
             &= 28429 / 161600 \cdot 100\% \\
             &= 18.59 \%.
 \end{split}
\end{equation}

\subsection{Economic conclusions}
After analyzing the given project from economical point of view, several conclusions were made. First of all, making such a research gave the possibility to understand better which are the strengths  and the weaknesses of the system and where additional attention should be paid in order to avoid unforeseeable external factors that could create unexpected problems.Several other indexes were computed such as direct and indirect expenses, tangible and intangible resources, worker remuneration, etc. As an observation, the biggest expense would be the salaries paid to employees and future improvements for maintaining the application up to date. Now, the next step is to find suitable clients, that would use the application  and, as a result, would suggest different improvements for the good functioning of the system. Finding at least one university for implementation purpose would represent a big step into the future of the application, guaranteeing an economic profit and great perspectives for upcoming versions.
\clearpage