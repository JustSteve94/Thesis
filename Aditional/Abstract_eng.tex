\section*{Abstract}
The thesis named \textbf{Designing Moldavian Government Websites on Drupal Platform}, presented by student Sîrbu Ștefan as a Bachelor project, was developed at the Technical University of Moldova. It is written in English and contains \pageref{LastPage} pages, \totaltables\ tables, \totalfigures\ figures, 16 listings and 13 references. The thesis consists of a list of figures, list of tables, introduction, four chapters, conclusion, and references list.

The main objective of the current document is to solve a real problem in the local academic structures. It involves processing the common semester timetables and generating individual schedules for each type of user.  There are three types of users defined in the application: teachers, students and administrators. The activities for each of them are defined as follows: the administrators are in charge of introducing the timetables in the system at the beginning of a new semester. There are more administrators allowed into the system, so that each of them could be responsible for a certain amount of data to be inserted. The students access the application via a registration form, which allows logging in the system. During registration, the student will be asked to select the membership group, guarantying this way the generation of the proper timetable for he/she. Teachers accounts, on the other side, are created by the administrators for a good structuring of product's components.

As first attempt, the product is proposed as a prototype that would represents a small change for the academic infrastructures, while the results are consistently improved. The application was built to be easy to implement in the university system, reliable at use and implies minimal effort to maintain. The system is independent of universities, meaning that it can be used by any local university for its own purposes.

The four chapters that compose the current report include the analysis of the encountered problem, the possible solutions for it and some additional hints for new features implementation. There are several addressed points of view that describe the current proposed solution. A careful review was performed, in order to emphasize all the strong sides of the applications and the components that need additional attention. Also, considering the current functionality provided by the application, some future directions for development were established. In order to ensure a plausible business plan for the stage when the product is released on the market, an economical research was accomplished. Analyzing the product from economical point of view offered a general idea about several financial indicators, which are crucial for the lifetime of the application.

This document is intended for readers with technical background.


