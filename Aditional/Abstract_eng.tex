\section*{Abstract}
The thesis named \textbf{Designing Moldavian Government Websites on Drupal Platform}, presented by student Sîrbu Ștefan as a Bachelor project, was developed at the Technical University of Moldova. It is written in English and contains \pageref{LastPage} pages, \totaltables\ tables, \totalfigures\ figures, 16 listings and 13 references. The thesis consists of a list of figures, list of tables, introduction, four chapters, conclusion, and references list.

Developing a website from zero is dificult job that requires a lot of time, but using a CMS simplifies your job as web developer. The main objective of the current document is to save developers time by using a Drupal distribution that will meet Moldavian Government website guidelines. This distribution will save your time and effort by allowing to move quickly to create a seamless website experience with distribution’s ready to go features. From downloading to hosting, from transferring content to customising and organising your Government website, you’ll quickly speed ahead. This distribution was refined to meet the specific requirements of the Moldavian Government, from perspectives of features this project is unique. Focus your time and effort on creating engaging content and customised features. Because your Government Drupal site comes fully loaded with a core set of user interface elements, functionality and features these can be reused as the basis for any new Moldavian Government website. For small Government websites, the majority of requirements have been met. Once it’s installed, you’re ready to go. For larger, more complex websites, you will have all the tools you need for customising and enhancing the features on your website.

The four chapters that compose the current report include the analysis of the encountered problem, the possible solutions for it and some additional hints for new features implementation. There are several addressed points of view that describe the current proposed solution. A careful review was performed, in order to emphasize all the strong sides of the applications and the components that need additional attention. Also, considering the current functionality provided by the application, some future directions for development were established. In order to ensure a plausible business plan for the stage when the product is released on the market, an economical research was accomplished. Analyzing the product from economical point of view offered a general idea about several financial indicators, which are crucial for the lifetime of the application.

This document is intended for readers with technical background.


