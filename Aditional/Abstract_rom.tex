\section*{Rezumat}
Teza de licență cu numele \textbf{Proiectarea Site-urilor Guvernamentale a Republicii Moldova pe Platforma Drupal}, prezentată de studenta Sîrbu Ștefan în calitate de proiect de Licență, a fost dezvoltată în cadrul Universității Tehnice din Moldova. Aceasta este scrisă în limba engleză și conține  \pageref{LastPage} pagini, \totaltables\ tabele, \totalfigures\ figuri, 16 listări de cod și 13 referințe. Teza consistă dintr-o listă de figuri, o listă de tabele, introducere, patru capitole, concluzie și bibliografie.

Obiectivul principal al lucrării curente este să rezolve o problemă reală apărută în sistemul academic local. Aceasta implică procesarea orarului semestrial general și crearea unor orare individuale pentru fiecare tip de utilizator. În cadrul aplicației sunt definiți trei tipuri de utilizatori: profesori, studenți și administratori. Activitățile pentru fiecare în parte sunt definite astfel: administratorii sunt responsabili de introducerea orarelor în sistem la începutul fiecărui semestru. Aplicația permite existența mai multor administratori, astfel încît fiecare dintre ei să fie responsabili de un anumit volum de date introduse. Studenții accesează aplicația prin forma de înregistrare, care permite logarea în sistem. În timpul înregistrării, studentul va fi rugat să selecteze grupa sa de apartenență, garantînd astfel generarea corectă a orarului pentru el/ea. Din alt punct de vedere, conturile profesorilor sunt create de administratori pentru o mai bună structurare a componentelor ce fac parte din produs.

Ca o primă încercare, produsul este propus ca un prototip care ar reprezenta o schimbare minoră în infrastructura academică, pe cînd rezultatele ar fi în mod vizibil îmbunătățite. Aplicația a fost dezvolată în așa mod încît să fie ușor implementabilă în sistemul universitar, sigură la utilizare, iar mentenanța - să fie ușor realizabilă. Sistemul este independent de universitate, acest lucru însemnînd că poate fi utilizat de orice universitate locală, pentru scopurile personale ale acesteia. 

Cele patru capitole care compun raportul curent includ analiza problemei întîmpinate, soluții posibile pentru aceasta și cîteva sugestii adiționale pentru implementarea unor noi caracteristici ale aplicației. Deasemenea, aplicația este analizată din cîteva puncte de vedere distincte și concrete, menite să descrie soluția curent propusă. O revizie minuțioasă a fost îndeplinită pentru a sublinia toate punctele forte ale aplicației și, totodată, componentele care necesită atenție adițională. Deasemenea, luînd în considerație funcționalitatea curentă a aplicației, cîteva direcții pentru viitoarea dezvoltare au fost stabilite. Pentru a asigura un business plan plauzibil pentru momentul în care produsul va fi lansat pe piață, o investigare economică a fost realizată. Analizînd produsul din punct de vedere economic, a fost posibila structurarea unei  idei generală asupra cîtorva indicatori financiari, care sunt cruciali pentru durata de viață a proiectului.

Acest document este destinat cititorilor specializați în domeniul tehnic.



