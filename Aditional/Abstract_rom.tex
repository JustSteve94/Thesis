\section*{Rezumat}
Teza de licență cu numele \textbf{Proiectarea Site-urilor Guvernamentale a Republicii Moldova pe Platforma Drupal}, prezentată de studenta Sîrbu Ștefan în calitate de proiect de Licență, a fost dezvoltată în cadrul Universității Tehnice din Moldova. Aceasta este scrisă în limba engleză și conține  \pageref{LastPage} pagini, \totaltables\ tabele, \totalfigures\ figuri, 16 listări de cod și 13 referințe. Teza consistă dintr-o listă de figuri, o listă de tabele, introducere, patru capitole, concluzie și bibliografie.

Dezvoltarea site-urilor web de la zero este un lucru dificil care necisită mult timp, doar că utilizînd un CMS simplifică lucrul tău ca dezvoltator web. Obiectivul principal al lucrării curente este economisească timpul dezvoltatorului prin utilizarea a distribuției Drupal care va satisface cerințele a Guvernului Republicii Moldova pentru site-urile web. Această distribuție vă va economisi timpul și efortul, permițându-vă să vă mișcați rapid pentru a crea o experiență fără probleme a site-ului cu funcțiile disponibile pentru distribuție. De la descărcarea la găzduire, de la transferarea conținutului la personalizarea și organizarea site-ului web al Guvernului, veți accelera repede. Această distribuție a fost îmbunătățită pentru a îndeplini cerințele specifice ale Guvernului Republicii Moldova, din perspectiva caracteristicilor acest proiect este unic. Concentrați-vă timpul și efortul asupra creării unui conținut captivant și a unor funcții personalizate. Deoarece site-ul dvs. guvernamental Drupal este complet încărcat cu un set de elemente de interfață cu utilizatorul, funcționalități și caracteristici, acestea pot fi refolosite ca bază pentru orice nou site Web al Guvernului Moldovei. Pentru site-urile guvernamentale mici, majoritatea cerințelor au fost îndeplinite. Odată ce este instalat, sunteți gata să plecați. Pentru site-uri mai mari, mai complexe, veți avea toate instrumentele de care aveți nevoie pentru personalizarea și îmbunătățirea funcțiilor pe site-ul dvs. web.

Cele patru capitole care compun raportul curent includ analiza problemei întîmpinate, soluții posibile pentru aceasta și cîteva sugestii adiționale pentru implementarea unor noi caracteristici ale aplicației. Deasemenea, aplicația este analizată din cîteva puncte de vedere distincte și concrete, menite să descrie soluția curent propusă. O revizie minuțioasă a fost îndeplinită pentru a sublinia toate punctele forte ale aplicației și, totodată, componentele care necesită atenție adițională. Deasemenea, luînd în considerație funcționalitatea curentă a aplicației, cîteva direcții pentru viitoarea dezvoltare au fost stabilite. Pentru a asigura un business plan plauzibil pentru momentul în care produsul va fi lansat pe piață, o investigare economică a fost realizată. Analizînd produsul din punct de vedere economic, a fost posibila structurarea unei  idei generală asupra cîtorva indicatori financiari, care sunt cruciali pentru durata de viață a proiectului.

Acest document este destinat cititorilor specializați în domeniul tehnic.



