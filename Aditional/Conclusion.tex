\section*{Conclusions}
Elaborating the inClass project, several steps were accomplished first. Before the actual development was started, the main goal was to understand entirely the problem that needed to be solved, by providing an optimal solution to it. Nowadays, we are living in the era of technological progress, where machines get to do the most part of the jobs people once used to do entirely. That's why, it is vital to sustain the every day life improvements human kind gets while creating software, instead of using the ancient approach of doing everything individually.  This is a thing that our country understood and during the last few years, more and more state structures are continuously developing their foundations, by providing reliable, modern technologies for the every day use. 

The current application refers to a specific type of state structures, that are local universities. These entities are providing a high rate of technological progress during the last years, thing that encourages the stuff and also the students and teachers. Given the fact that it has been seized a continuously growing interest in determining people to work with software, now becomes much more realistic to reorganize the academic structures. This means that people already reached the moment when are ready to bring software products in universities in order to optimize the routine processes which they are responsible for accomplishing.  This is a big step that has to be adjusted properly to the university's needs. 

inClass is the software product that comes with a project in development and a prototype to fix an existing problem in universities. At the same time, it represents a small change for the academic infrastructure, given the fact that is easy to implement, use and maintain, while the results are consistently improved.The application was implemented as a web platform, assuming that universities don't have the problem of free connection to the internet anymore. So, in order to access the application, it is enough to own a smartphone and Wi-Fi. It represents the perfect solution for avoiding the general timetable, which is so uncomfortable to reach and use.

For a better representation of the application, a full analysis over the system was done first. It consists of UML diagrams, which have the purpose of visualizing the application from different points of view. After this schematic overview of the system is done, there remain no pending questions about the functionality provided by the application. Once things get clear, it is time to start implementing those functionalities. The technology used for developing the application is a powerful one. Ruby on Rails is one of the most high rated frameworks at the current moment, which allows the developer to implement various functionality with a minimum of written code. Also, it allows the use of gems, which simplify the task even more. 

Finally, assuming that the project is finished, it is necessary to provide an user guide, for easing the tasks of the actual users. There are several moments that need to be paid attention to, moments that were clearly specified in the content of the current document. The main thing to be mentioned is that the application allows 3 types of users: administrator, teachers and student. In order to use the application it is necessary to register in the system. Registration is specific for each type of user. 

At the moment, the provided solution is unique on the local market, thing that is convenient from economical point of view. There are necessary less expenses to take into account in order to finish the project successfully. The less expenses are, the less the market price will be, which makes the software available for use for local universities. In terms of economical analysis, the numbers were pretty impressive. If computing all the possible expenses that could occur while developing the project with some more improvements, the team would also need to find sufficient clients, that would guarantee a low price of the product and more than 18 \% as profit.

Also, is a good idea of coming with some future possible improvements for the application. The most recent improvement would definitely be the adjustments of the application to support sorting timetables by current weak, meaning that the even and odd weeks would be take in account. Of course, this is a requirement that is not valid for all universities, but it is still very important for the ones that have different schedules for even and odd weeks. Another improvement could be done for the student/teacher side of the application. At the current moment, the teachers and students are allowed only to visualize data. Assuming that the current project it is a prototype, this is acceptable. Still, the idea of having several tools available for checking the current status of a certain day or semester, by using progress bars and other visual components, would enlarge the boundaries of activity for these types of users. Another feature to be implemented could be the possibility of interaction between student and teachers, while discussing some updates of the courses via a special page for announcements. Many other improvements can be done once the universities get to the point when they implement the system as part of their infrastructures. 

\clearpage