\section*{Introduction}
As a retrospective on student's life now and then, it becomes clear that things have changed during the time. Nowadays, living became comparable more intense than a couple of decades ago, generally thanks to the revolution of technologies. People are dynamic, innovations appear everyday and it is necessary to keep track on them. At a certain moment, all this agitation can result in a loss of control when speaking about the management of personal schedules. A common problem encountered in universities is the fact that students become easily distracted. During an usual academic day, a student has to face several courses with minimal breaks between them. Usually, in the attempt of focusing on other activities that result in more than the admitted limit of student's break time, it is not seized how the activities in cause are overlapping with the studying schedule. At this point, the student's productivity in the boundaries of the university become affected. So, here appears the need of having more control over the university's timetable by avoiding physical presence for checking courses status and replacing it with something else. 

At the moment, the only schedule available for reminding which is the flow of the courses for a given day is usually situated in the main hall of the university and generally looks like a a printed version of an Excel document. This fact becomes quite inconvenient, since our country is making big efforts for growing up from manual systems to the computerized ones. As a quick fix for avoiding such a situation, it is necessary to make studying schedules easier to reach by developing an application that would guarantee access to it via smartphones. Also, given the fact that internet connection via Wi-Fi on the university's territories is a prerogative nowadays, a web application would be the most suitable solution. 

It is worth mentioning that such systems have been developed and implemented by abroad countries for a good period of time already, so it's time for Republic of Moldova to make some changes in the current academic system, too. The truth is that the systems mentioned above are even more complex from functional and structural point of view. They are build as an entire component having automatic generation of studying schedule, detailed description of courses, online assignments, course materials and others. Such a platform is quite similar with the application known by local students as Moodle, just offering more flexibility and area of activity for its users.

The transition to such a complex platform could have unwanted consequences, that's why it's preferable to try some kind of prototype for a better observation and understanding of local student's needs. The product proposed in this document represents that prototype mentioned above, that has as main goal to test the current target group, which is represented by teachers and students and evaluate their attitude regarding the system. It is trivial to understand if the people involved are willing to accept this changes because that would eventually lead to a bigger step of introducing a complex platform, as one described above. 

inClass is the product designed for solving a real problem in the local academic structures. It involves processing the common semester timetables and generating individual schedules for each type of user. There are three types of users defined in the application: teachers, students and administrators. The administrators are the persons responsible for introducing the raw data into the system, while the logic of the application is in charge with the rest. Given the fact that there are more administrators allowed into the system, each of them would be responsible for a certain amount of data to be inserted. The system is independent of universities, meaning that it can be used by any local university for its own purposes. All the entities are created by administrators, so a lot of space is allowed to decide the format of usage for the application. 

For students, to access the application, it is necessary to complete a registration form and log in with those credentials after the process is completed. During registration, the student will be asked to select the membership group, guarantying this way the generation of the proper timetable for he/she. Teachers accounts, on the other side, are created by the administrator in charge in order to avoid disorders in the system. After completing registration, the admin is responsible of securely hand out the credentials to the teacher for which the account was created. It is important to mention that each introduced course has a label specified, which is the teacher's name. So, the timetable for teachers is created depending on the courses having the proper labels. Also, it is well known the fact that one teacher can have courses with students from different faculties, that's why the product in discussion is set up to display the information about the faculties where courses take place, despite the other details. For general description purpose, the system is able to display customized information about courses, depending on the type of user. More particular features will be discussed in the next chapters. This was the general description of the product. Detailed information is available in the further pages.

The current thesis consists of 4 chapters. Chapter 1 explains the system analysis, together with the general overview about the final product. A market research has been done, in order to define advantages/disadvantages of using the application and provide a better understanding over the resulting system. Chapter 2 contains a more complex description of the system, from architectural point of view. The system is examined from different approaches, using UML diagrams. Chapter 3 contains a brief explanation regarding the technologies used for the implementation process and how those technologies were linked to the application. Also, the system's basic features are discussed from technical point of view, motivating the choice for that specific implementation. Chapter 4 represents a review over the economical position of the product. It is presented an estimation of costs for developing the project and some economical prediction for the future use.