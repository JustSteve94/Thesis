\section*{Introduction}
In our technology era, Internet is one of the most useful tools where big amounts of data circulates through the network which covers the whole world. We spend our time rushing after this information cause knowledge is power and data is money. Nowadays website are the efficient way to present and spread information to the maximum number of people in the world. By using web browsers you can access the rendered information from a webpage which basically represents a structure from blocks of media or text content.

Websites, which represents a mix between hypermedia and information systems, from users point of view websites have relatively simple architecture but from developers point of view websites have a more complex dynamic architecture. Websites need to respond to an unlimited number of heterogeneous user’s request, privacy and security concerns, up to date information sources.

Any organisation in the world need a website alike the government institutions need one. Using a website an organisation is promoting itself by providing information about them, such as who they are, what they do and how they do. The goal of my thesis is to develop a web platform for all governmental institutions.

In the first chapter we will discus about domain analysis of the implemented product and we will see the difference and similarity between this product and others. Of course we will talk about why do we need this system. Here we will define notions which will be used mostly in thesis and why they are important from various perspectives. 

After comes system design and how it was planned to be. Using various visual representations will help you understanding the structure and relationships between system components.

In the third chapter you will find out how the product was developed, the languages and techniques were used to create the product. Also you will see pieces of system’s source code, what problems were encountered developing the product and their solutions. Here you will perceive what technologies was used and why, further you will learn the implementation of their futures.

In continue we will explain how the product can be used. We will show up installing, configuration steps. This chapter is a user guide for the product, that can be used by a non technical person. User guide includes screenshot images which describes step by step how the user can use the product.

The last chapter but not the least is about a business plan for developing the product. This plan represents an economic representation for the project which covers how much time we need to accomplish projects goals, who is involved in project and how much they will be paid. Mostly this chapter is about projects profitability for this product.